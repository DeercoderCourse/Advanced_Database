\documentclass{article}
\author{Chang Liu ~\\ chang\_liu@student.uml.edu}
\title{91.673 Advanced Database Systems Homework1}

\begin{document}

\maketitle



\noindent \textbf{Problem1.} We flip a fair coin ten times. Find the probability of the following events.

(a) The number of heads and the number of tails are equal.

(b) There are more heads than tails.

(c) The ith flip and the (11 – i)th flip are the same for i = 1, . . . , 5. 

\textbf{Solution:}~\\

(a) The number of heads equals to the number of tails when both of them are \textbf{5}. So the probability is that when the head is 5(and the tail is also 5 in this case), so the probability is that:

$$C_{10}^5 * (1/2)^5 * (1/2)^{(10-5)} = 0.246$$



\noindent \textbf{Problem2.} Consider the following game, played with three standard six-sided dice. If the player ends
with all three dice showing the same number, she wins. The player starts by rolling all three dice. After
this first roll, the player can select any one, two, or all of the three dice and re-roll them. After this second
roll, the player can again select any of the three dice and re-roll them one final time. For the following
questions, assume that the player used the following optimal strategy: if all three dice match, the player
stops and winds; if two dice match, the player re-rolls the die that does not match; and if no dice match,
the player re-rolls them all.

(a) Find the probability that all three dice show the same number on the first roll.

(b) Find the probability that exactly two of the three dice show the same number after the first roll.

(c) Find the probability that the player wins, conditioned on exactly two of the three dice showing the
same number after the first roll.
 

\noindent \textbf{Problem3.} We have a function F: {0,...,n-1} → {0,...,m-1}. We know that, for 0 ≤ x, y ≤ n − 1, F((x +
y) mod n) = (F(x) + F(y)) mod m. The only way we have for evaluating F is to use a lookup table that
stores the values of F. Unfortunately, an Evil Adversary has changed the value of 1/5 of the table entries
when we were not looking.

Describe a simple randomized algorithm that, given an input z, outputs a value that equals F(z) with
probability at least 1/2. Your algorithm should work for every value of z, regardless of what values the
Adversary changed. Your algorithm should use as few lookups and as little computation as possible.


Suppose I allow you to repeat your initial algorithm three times. What should you do in this case, and
what is the probability that your enhanced algorithm returns the correct answer?



\noindent \textbf{Problem4.} . Suppose we roll a fair k-sided die with the numbers 1 through k on the die's faces. If X is the
number that appears, what is E[X]?



\noindent \textbf{Problem5.} A monkey types on a 26-letter keyboard that has lowercase letters only. Each letter is chosen
independently and uniformly at random from the alphabet. If the monkey types 1,000,000 letters, what is
the expected number of times that sequence "proof" appears?



\end{document}
