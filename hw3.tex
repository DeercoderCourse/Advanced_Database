\documentclass{article}
\usepackage{amsmath}
\author{Chang Liu ~\\ chang\_liu@student.uml.edu}
\title{91.673 Advanced Database Systems Homework3}
\begin{document}

\maketitle

\section{Problem 1}

\textbf{Exercise 4.3.1}: For the situation of our running example (8 billion bits, 1 billion members of the set S), calculate the false-positive rate if we use three hash functions? What if we use four hash functions?


\section{Problem 2}

\textbf{Exercise 4.4.1}: Suppose our stream consists of the integers 3, 1, 4, 1, 5, 9, 2, 6, 5. Our hash functions will all be o f the form h(x)=ax+b mod 32 for some a and b. You should treat the result as a 5-bit binary integer. Determine the tail length for each stream element and the resulting estimate of the number of distinct elements if the hash function is:

(a) h(x)=2x+1 mod32.

(b) h(x)=3x+7 mod32.

(c) h(x) = 4x mod 32.



\section{Problem 3}

\textbf{Exercise 4.6.1}: Suppose the window is as shown in Fig. 4.2. Estimate the number of 1’s the the last k positions, for k = (a) 5 (b) 15. In each case, how far off the correct value is your estimate?


\end{document}

